\documentclass[12pt]{nature-mod}
\usepackage{graphicx,color,multirow,bm,braket,amsmath,soul,epstopdf}
\def\bk{{\bf k}}
\def\bq{{\bf q}}
\def\ve{\varepsilon}
\def\e{\epsilon}

\begin{document}

%\title{Origin of the crossover from polarons to Fermi liquids in transition metal oxides
%\vspace{0.3cm}\\Supplementary Information}
%\author{Carla Verdi, Fabio Caruso, Feliciano Giustino}
%\email{feliciano.giustino@materials.ox.ac.uk}

%\maketitle

%\vspace{0.2cm} 
%\begin{affiliations} 
%\item[] Department of Materials, University of Oxford, Parks Road, Oxford OX1 3PH, United Kingdom
%\end{affiliations} 

 \begin{figure}
 \begin{center}
 \includegraphics[width=0.95\textwidth]{FIGURES1.png}
 \end{center}
 \textbf{\bf Supplementary Figure 1 $\,\bm|\,$ Crystal structure and phonon dispersions of anatase 
 TiO$_{\bm2}$.} ({\bf a})~Ball-and-stick model of the tetragonal unit cell of anatase TiO$_2$. Ti 
 atoms are in gray and O atoms are in red. The lattice vectors and crystallographic directions referred
 to in the main text are indicated. ({\bf b})~Calculated phonon dispersion relations of anatase
 TiO$_2$ along high-symmetry directions of the Brillouin zone. The $E_u$ and $A_{2u}$ phonons discussed 
 in the main text are indicated. The $A_{2u}$ phonon is infrared active only along the $c$ axis, 
 accordingly it is found to give a smaller contribution to the total Fr\"ohlich coupling after 
 integrating over all phonon momenta.
 \end{figure}
 
 \clearpage
 
 \begin{figure}
 \begin{center}
 \hspace*{-8pt} \includegraphics[width=0.95\textwidth]{FIGURES2.png}
 \end{center}
 \textbf{Supplementary Figure 2 $\,\bm|\,$ Non-adiabatic electron-phonon matrix elements in anatase 
 TiO$_{\bm2}$.} ({\bf a}) Lindhard dielectric function associated with doped carriers in anatase TiO$_2$ 
 at different values of $\bq$. For illustration we consider the highest doping level, 
 $3.5\times10^{20}$~cm$^{-3}$.
 ({\bf b})~Non-adiabatic electron-phonon matrix elements $g^{\rm NA}_{mn\nu}(\bk,\bq)$ corresponding 
 to the LO E$_u$ phonon of anatase TiO$_2$, as a function of doping level. We set $n\bk$ to the 
 bottom of the conduction band. The electron-phonon matrix element becomes weaker at high doping.
 ({\bf c})~Comparison between the density of states (DOS) near the conduction band bottom 
 calculated from first principles (red) and modelled with a parabolic band (blue): 
 $\mbox{DOS}(E)=V/(2\pi^2)\,(2m_{\rm b}/\hbar^2)^{3/2}\sqrt{E}$, where $V$ is the unit cell 
 volume of anatase. The \emph{ab initio} DOS was computed using Wannier interpolation to obtain the 
 electronic energies on a randomly generated dense $\bk$ grid and with a Gaussian smearing of 5~meV. 
 The gray vertical line indicates the energy corresponding to the Fermi level at the highest doping. 
 The parabolic DOS starts deviating significantly from the calculated one for energies above 0.2~eV. 
 \end{figure}
 
\clearpage

 \begin{figure}
 \begin{center}
 \hspace*{-8pt} \includegraphics[width=0.85\textwidth]{FIGURES3.png}
 \end{center}
 \textbf{Supplementary Figure 3 $\,\bm|\,$ Spectral function in the transition regime.}
 ({\bf a})~Spectral function of anatase TiO$_2$ calculated for the doping level $1\times10^{20}$~cm$^{-3}$. 
 Gaussian masks of widths 25~meV and 0.015~\AA$^{-1}$ were applied as in Fig.~1 of the main text. 
 ({\bf b})~Band structure extracted from~{\bf a} (blue lines) together with the bare band (red line). 
 For this doping one satellite is still visible but the mass renormalization parameter is decreasing 
 ($\lambda=0.34$ as reported in the main text).
 \end{figure}
 
\clearpage

 \begin{figure}
 \begin{center}
 \hspace*{-8pt} \includegraphics[width=0.95\textwidth]{FIGURES4.png}
 \end{center}
 \textbf{Supplementary Figure 4 $\,\bm|\,$ Impact of the electron lifetime on the spectral properties.}
 Spectral function of anatase TiO$_2$ calculated for the doping level $3.5\times10^{20}$~cm$^{-3}$ 
 using $\hbar/\tau=25$ ({\bf a}), 55 ({\bf b}) and 75 meV ({\bf c}) to compute the electronic screening 
 (see Methods). The spectral features are virtually unchanged. We recomputed the mass renormalization 
 parameter $\lambda$ as in the main text which gives $\lambda=0.19$, 0.20 and 0.22 for increasing 
 broadening. The same analysis for the doping concentration $5\times10^{18}$~cm$^{-3}$ yields 
 $\lambda=0.77$, 0.73 and 0.72 for the same broadenings. These values fall within 10\% of 
 the results presented in the main text. 
 \end{figure}

\clearpage

\textbf{Supplementary Note 1 $\,\bm|\,$ Cumulant expansion}

The spectral function is related to the imaginary part of the one-electron retarded Green's function by:
  \begin{equation} \label{eq.specfun}
  A(\bk,\omega)=-\frac{1}{\pi} {\sum}_n {\rm{Im}}\,G_{n\bk}(\omega).
  \end{equation}
In the cumulant expansion the Green's function is obtained in the time domain, starting from the
the interaction picture\cite{Langreth1970,Hedin1980,Aryasetiawan1996,Gumhalter2016}:
  \begin{equation}
  G_{n\bk}(t)=i\,\exp\left[-i\ve_{n\bk}t/\hbar+C_{n\bk}(t)\right],
  \end{equation}
where $C_{n\bk}(t)$ is the cumulant function. By taking the Fourier transform of this expression
to the frequency domain and inserting it in Supplementary Eq.~\eqref{eq.specfun} we obtain Eq.~(1) 
of the main text. 
In order to obtain an expression for the cumulant which is amenable to computation, it is customary 
to expand the exponential in powers of the cumulant: $G_{n\bk} = G_{n\bk}^0 \left[1 + C_{n\bk} + 
C_{n\bk}^2/2+\cdots\right]$, where $G_{n\bk}^0(t) = i\,\exp\left(-i\ve_{n\bk}t/\hbar\right)$ is
the Green's function in absence of electron-phonon interactions. Alternatively, the Green's function
can be obtained from the Dyson equation $G_{n\bk} = G_{n\bk}^0 + G_{n\bk}^0 \varSigma_{n\bk} G^0_{n\bk}
+ G_{n\bk}^0 \varSigma_{n\bk} G^0_{n\bk} \varSigma_{n\bk} G^0_{n\bk} + \cdots$. By comparing these
expansions term-by-term one finds an explicit expression for the cumulant function, which is
given in the Methods and is reproduced here for completeness:
  \begin{equation}\label{eq.cum-sigma}
  C_{n\bk}(t)=\frac{1}{\pi \hbar}\int_{-\infty}^{\infty}{\rm d}\omega \,
  \frac{{\rm Im}\,\varSigma_{n\bk}(\ve_{n\bk}/\hbar-\omega)}{(\omega-i\eta)^2}\,e^{(i\omega+\eta)t}
  \,\theta(\ve_{\rm F}-\ve_{n\bk}+\hbar\omega).
  \end{equation}
In practical calculations the exact self-energy $\varSigma_{n\bk}$ is replaced by the best available 
approximation, which is the Migdal expression given in Eq.~(2) of the main text. A rigorous derivation
of the cumulant expansion and a discussion of its advantages and limitations can be found in 
Supplementary Refs.~\citenum{Gumhalter2016,Zhou2015}. In particular, in 
Supplementary Ref.~\citenum{Gumhalter2016} it is shown that the choice of seeding 
the self-energy calculated with the first non-crossing diagram also 
guarantees that no overcounting of correlated higher-order contributions is introduced in the theory. 
The inclusion of crossing diagrams in the evaluation of the Green's function results from the time 
orderings of the $t$ variables in the cumulant expansion.

The spectral function given in Eq.~(1) of the main text yields multiple bosonic satellites, one
for each term in the Taylor expansion of $\exp [C_{n\bk}(t)]$. A convenient expression for the
case of a single satellite was derived in Supplementary Ref.~\citenum{Aryasetiawan1996}. 
In this work we extended the method of Supplementary Ref.~\citenum{Aryasetiawan1996} 
to the case of multiple satellites. Following Supplementary 
Ref.~\citenum{Aryasetiawan1996}, we write the spectral function as:
 \begin{equation} \label{eq.spec.cum}
 A(\bk,\omega)={\sum}_n 
  \left[A_{n\bk}^{\rm{QP}}(\omega)+A_{n\bk}^{\rm{QP}}(\omega)\ast A_{n\bk}^{\rm S1}(\omega)
    +A_{n\bk}^{\rm{QP}}(\omega)\ast A_{n\bk}^{\rm S2}(\omega) + \cdots\right],
 \end{equation}
where $\ast$ indicates the convolution. In the last expression
the quasiparticle contribution $A_{n\bk}^{\rm{QP}}(\omega)$ is defined as:
 \begin{equation} \label{eq.Aqp}
  A_{n\bk}^{\rm{QP}}(\omega)= \frac{2}{\pi}\frac{|{\rm Im}\varSigma_{n\bk}(\ve_{n\bk})|}
    {[\hbar\omega-\ve_{n\bk}-{\rm Re}\varSigma_{n\bk}(\ve_{n\bk})]^2+[{\rm Im}\varSigma_{n\bk}(\ve_{n\bk})]^2} ,
 \end{equation}
and the satellite contributions associated with one-phonon and two-phonon processes are:
\begin{equation} \label{eq.Acum}
 A_{n\bk}^{\rm S1}(\omega) = \int_{-\infty}^{\infty} dt\,e^{i\omega t}  C_{n\bk}(t),
    \qquad A_{n\bk}^{\rm S2}(\omega) =  
  \int_{-\infty}^{\infty} dt\,e^{i\omega t}  \frac{1}{2}C_{n\bk}(t)C_{n\bk}(t) \,.
\end{equation}
The function $A^{\rm S1}$ can be written in terms of the electron-phonon self-energy by combining 
Supplementary Eqs.~\eqref{eq.Acum} and \eqref{eq.cum-sigma} and carrying out the Fourier 
transform. This step was performed in Supplementary Ref.~\citenum{Aryasetiawan1996}: 
 \begin{equation}\label{eq-spectrum2}
  A_{n\bk}^{\rm S1}(\omega)  
  =  \frac{\beta_{n{\bf k}}(\omega) - \!\beta_{n{\bf k}}(\varepsilon_{n{\bf k}}) - 
  \!(\omega-\varepsilon_{n{\bf k}}/\hbar)\!\left. 
  \displaystyle\frac{\partial \beta_{n{\bf k}}}{\partial \omega}
    \right|_{\varepsilon_{n{\bf k}}/\hbar}}
  {(\hbar\omega-\varepsilon_{n{\bf k}})^2}, 
  \end{equation}
 with
 \begin{equation}
  \beta_{n{\bf k}}(\omega) = \frac{1}{\pi}{\rm Im}\,\varSigma_{n{\bf k}}(\varepsilon_{n{\bf k}}/\hbar-\omega)
\theta(\ve_{\rm F}/\hbar-\omega).
 \end{equation}
In order to obtain the contribution of the second satellite, we note that $A_{n\bk}^{\rm S1}(\omega)$
is the Fourier transform of $C_{n\bk}(t)$, therefore the expression for $A_{n\bk}^{\rm S2}(\omega)$
in Supplementary Eq.~\eqref{eq.Acum} can be rewritten using the convolution theorem:
  \begin{equation} \label{eq.s2}
  A_{n\bk}^{\rm S2} = \frac{1}{2} A_{n\bk}^{\rm S1} \ast A_{n\bk}^{\rm S1}.
  \end{equation}
By combining Supplementary Eqs.~\eqref{eq.s2} and \eqref{eq.spec.cum}
we obtain the expression used in our calculations:
  \begin{equation} \label{eq.ourcalc}
  A(\bk,\omega)={\sum}_n \left[ A_{n\bk}^{\rm{QP}}(\omega)+ A_{n\bk}^{\rm S1}(\omega) 
  \ast A_{n\bk}^{\rm{QP}}(\omega) + \frac{1}{2} A_{n\bk}^{\rm S1}(\omega) \ast  
  A_{n\bk}^{\rm S1}(\omega) \ast A_{n\bk}^{\rm QP}(\omega) +\cdots\right].
  \end{equation}
From Supplementary Eq.~\eqref{eq.ourcalc} we can extrapolate the general expression for 
the case of many satellites:
  \begin{equation}\label{eq.exponent}
  A(\bk,\omega)={\sum}_n \left[ 1+
  A_{n\bk}^{\rm S1}(\omega) \ast 
  + \frac{1}{2} A_{n\bk}^{\rm S1}(\omega) \ast  A_{n\bk}^{\rm S1}(\omega) \ast + \cdots \right]
  A_{n\bk}^{\rm{QP}}(\omega)  = {\sum}_n \displaystyle e^{A_{n\bk}^{\rm S1}(\omega) \ast }
  A_{n\bk}^{\rm{QP}}(\omega).
  \end{equation}
When considering a drastically simplified model system consisting of dispersionless electrons
and an Einstein phonon spectrum, this last expression reduces to the well-known Lang-Firsov 
series of polaron satellites\cite{Langreth1970,Berciu2006}.
In fact, by setting $A^{\rm QP} = Z\delta(\omega)$ and $A^{\rm S1} =
\lambda\delta(\omega-\omega_{\rm ph})$, Supplementary Eq.~\eqref{eq.exponent} gives
$A(\omega) = Z \sum_{m=0}^\infty (\lambda^m/m!) \,\delta(\omega-m\,\omega_{\rm ph})$.
By further requiring the normalization of the spectral function, that is 
$\int\!A(\omega)\,{\rm d}\omega=1$, we obtain $Z=e^{-\lambda}$ and the Lang-Firsov 
expression is recovered.
In our {\it ab initio} calculations we truncate Supplementary Eq.~\eqref{eq.exponent} to 
the second order. In practice we first evaluate the quasiparticle and satellite contributions, and 
then we perform two successive numerical convolutions in order to obtain the two satellites which are 
seen in the experiments.

\vspace{0.5cm}
\textbf{Supplementary Note 2 $\,\bm|\,$ Mass enhancement and Fr\"ohlich coupling constant}

The mass enhancement parameters obtained from the quasiparticle bands along the $\Gamma\Sigma$,
$\Gamma$X, and $\Gamma$Z directions are $\lambda = 0.73$, 0.73, and 0.74 at the lowest doping; 
$\lambda=0.70$ in each direction at the intermediate doping; and $\lambda =0.20$, 0.20, and 0.18 
at the highest doping. The mass enhancement can also be calculated directly from the energy derivative 
of the real part of the electron self-energy at the Fermi level, $\lambda_{\bf k}=-\partial{\rm Re}
\varSigma_{\bf k}(\omega)/\partial\omega|_{\omega=\ve_{\rm F}}$\cite{Grimvall}. We checked that the values 
thus obtained fall within less than 10\% of the ones listed above, the average over the Fermi surface 
being \mbox{$\lambda=0.68$,} 0.65 and 0.19 for the three values of doping considered. This approach also 
validates our calculations of the contribution to $\lambda$ arising from different phonon modes presented 
in the analysis of Fig.~2a in the main text.

From the mass enhancement parameter it is also possible to obtain the quasiparticle strength $Z$ as 
$Z=1/(1+\lambda)$~\cite{Grimvall}. Using the values of $\lambda$ reported in Fig.~1g-i we obtain 
$Z=0.58$, 0.59, and 0.83 with increasing doping. Our calculated quasiparticle strength at intermediate 
doping overestimates the experimentally-determined value, $Z=0.36$~\cite{Moser2013}. We attribute this 
difference to extrinsic losses not accounted for in our calculations; these losses are known to transfer 
spectral weight to the satellites\cite{Hedin1985, Aryasetiawan1996, Kas2016}.

In polar materials it is customary to describe the coupling to a dispersionless LO phonon via the 
dimensionless Fr\"ohlich coupling constant $\alpha$:
  \begin{equation} \label{alpha.def}
  \alpha=\frac{e^2}{\hbar}\left(\frac{m_{\rm b}}{2\hbar\omega_{\rm LO}}\right)^{1/2}
  \left(\frac{1}{\epsilon_\infty}-\frac{1}{\epsilon_0}\right),
  \end{equation}
where $m_{\rm b}$ is the band mass, $\hbar\omega_{\rm LO}$ the energy of the LO phonon, $\epsilon_0$ 
and $\epsilon_\infty$ the static and high-frequency permittivities, respectively. We evaluate 
Supplementary Eq.~\eqref{alpha.def} using the experimental values of $\omega_{\rm LO}$, 
$\epsilon_0$, and $\epsilon_\infty$ from Supplementary Ref.~\citenum{Berger1997}. 
For the effective masses we use our DFT 
calculations since no accurate experimental values are available: $m_\perp=0.40\, m_{\rm e}$ and 
$m_\parallel=4.03\,m_{\rm e}$. By using these parameters in Supplementary Eq.~\eqref{alpha.def} 
we find the experimental Fr\"ohlich constants $\alpha_\perp=1.0$ and $\alpha_\parallel=3.4$, and their 
isotropic average $\alpha=1.8$. The Fr\"ohlich constant $\alpha$ can also be obtained from {\it ab initio} 
calculations of the electron-phonon matrix elements, following Supplementary Ref.~\citenum{Verdi2015}. 
In this case we calculate $\alpha_\perp=0.9$, $\alpha_\parallel=3.0$, and the isotropic average $\alpha=1.6$. 
These values are in excellent agreement with experiment, therefore our description of Fr\"ohlich 
coupling in anatase TiO$_2$ is expected to be highly accurate. We note that the anisotropy of the 
Fr\"ohlich coupling constant $\alpha_\perp$ and $\alpha_\parallel$ does not stem from anisotropic 
electron-phonon interactions, but rather from the strong anisotropy in the band masses, which enter 
$\alpha$ as seen in Supplementary Eq.~\eqref{alpha.def}.

For weak and intermediate couplings, the polaron effective mass is commonly estimated using 
$m^*= m_{\rm b}(1-\alpha/6)^{-1}$~\cite{Frohlich1954, Mishchenko2000}. While this procedure is
adequate for isotropic crystals, it cannot be used in the present case of TiO$_2$. In fact, if 
we use the isotropic average of the coupling constant, then we underestimate $m^*$ with respect 
to experiment\cite{Moser2013}. On the other hand, if we consider $\alpha_\perp$ and $\alpha_\parallel$ 
separately, then we obtain a strongly anisotropic mass enhancement, which is not consistent with our 
many-body calculations. These observations indicate that, in the case of anisotropic crystals, the 
Fr\"ohlich constant $\alpha$ should be used with caution.

\vspace{0.5cm}
\textbf{Supplementary Note 3 $\,\bm|\,$ Dielectric screening} 

In this work we calculate the additional screening arising from the charge carriers
using the random phase approximation (RPA) for the homogeneous electron gas, that is
the Lindhard screening. This choice is motivated by the fact that we must evaluate
the screening for millions of electron-phonon matrix elements, and explicit
{\it ab initio} calculations of the RPA screening for such a dense Brillouin zone
grid are not currently feasible. The use of the Lindhard model is justified by the fact that
the system under consideration lies in the high-density electron-gas limit.
To confirm this point we evaluate the Wigner-Seitz radius $r_s$ given by
$r_s=(4\pi na_0^{\ast3}/3)^{-1/3}$, where $n$ is
the doping density and $a_0^\ast=\hbar^2\epsilon_0/(e^2 m_{\rm b})$\cite{Mahan}.
Using the band effective
mass $m_{\rm b}$ and the static dielectric constant $\epsilon_0$ of anatase we obtain
$r_s=1.6$ for
the lowest doping level $5\times10^{18}$~cm$^{-3}$. This value is comparable or even smaller 
than what found in simple metals, therefore the use of the electron gas model to describe
doped carriers is justified.
Moreover, the electronic bands are to a good approximation 
parabolic in the energy range considered in this work;
this is seen by comparing the {\it ab initio} density of states with the parabolic
band model, Supplementary Fig.~2c. Therefore the use of a Lindhard function is 
justified, and we expect this choice to be very accurate in the present case.

\vspace{0.5cm}
\textbf{Supplementary Note 4 $\,\bm|\,$ Polaron wavefunction} 

In order to calculate the wavefunction of a polaron, we follow the approach of Supplementary 
Ref.~\citenum{Mahan} and express the many-body electron-phonon state using Rayleigh-Schr\"odinger 
perturbation theory. We consider a system at zero temperature and with a single electron added to the 
bottom of the conduction band. The resulting expression is:
  \begin{equation} \label{wfc.pert}
  \tilde{\psi}_{n\bk}({\bf r};\{ {\bm\tau}_{\kappa} \})=
  \psi_{n\bk}({\bf r})\ket{0_{\rm p}} + \frac{1}{\sqrt{N_\bq}}\sum_{m\nu\bq} \frac{
  g_{mn\nu}(\bk,\bq) \psi_{m\bk+\bq}({\bf r})} {\ve_{n\bk}-\ve_{m\bk+\bq}-\omega_{\bq\nu}}
  \,\hat a^\dagger_{-\bq\nu}\ket{0_{\rm p}},
  \end{equation}
where $\{ {\bm\tau}_{\kappa} \}$ are the nuclear coordinates, 
$\hat a^\dagger_{-\bq\nu}$ is a phonon creation operator, and $\ket{0_{\rm p}}$ is the ground 
state with no phonons. The Brillouin zone is discretized using a uniform grid with $N_{\bq}$ phonon
wavevectors. Since the atomic displacements are smaller than characteristic interatomic distances, 
we can simplify the above expression by replacing $\hat a^\dagger_{-\bq\nu}$ (which is a function 
of the normal mode coordinates) by its average over a given electron-phonon state\cite{Mahan}. 
In order to determine a lower bound to the polaron radius, we evaluate this expectation value by 
considering an electronic wavefunction localized at the center of the reference frame, ${\bf r}=0$. 
The result is:
  \begin{equation}
  \langle\hat a_{-\bq\nu}^\dagger\rangle= \frac{1}{\sqrt{N_\bq}}\sum_{m'} 
  \frac{ g^\ast_{m'n\nu}(\bk,\bq) } {\ve_{n\bk}-\ve_{m'\bk+\bq}-\omega_{\bq\nu}}.
  \end{equation}
By replacing $\langle\hat a_{-\bq\nu}^\dagger\rangle$ inside Supplementary Eq.~\eqref{wfc.pert} 
we have:
  \begin{equation}\label{eq.pol-wfc}
  \tilde{\psi}_{n\bk}({\bf r};\{ {\bm\tau}_{\kappa} \})= \frac{e^{i\bk\cdot{\bf r}}}{\sqrt{\varOmega}} 
  \left[ u_{n\bk}({\bf r}) + \frac{1}{N_\bq}\sum_{m\nu\bq} \frac{
  g_{mn\nu}(\bk,\bq) u_{m\bk+\bq}({\bf r})e^{i\bq\cdot{\bf r}} } {\ve_{n\bk}-\ve_{m\bk+\bq}-\omega_{\bq\nu}} 
  \sum_{m'} \frac{g^\ast_{m'n\nu}(\bk,\bq)} {\ve_{n\bk}-\ve_{m'\bk+\bq}-\omega_{\bq\nu}} \right] 
  \ket{0_{\rm p}} .
  \end{equation}
The expression for $\tilde{\psi}_{n\bk}$ given in the Methods was obtained from Supplementary 
Eq.~\eqref{eq.pol-wfc} by considering the normalization of the wavefunction and by retaining the long-wavelength 
part of the Fr\"ohlich vertex, so that $g_{mn\nu}(\bk,\bq)\rightarrow g_{mn\nu}(\bk,\bq)\,\delta_{mn}$. 
We already demonstrated that this approximation is very accurate for anatase TiO$_2$\cite{Verdi2015}. Since 
the main contribution to Supplementary Eq.~\eqref{wfc.pert} arises from 
long-wavelength optical phonons, we also replace $u_{m\bk+\bq}(\mathbf r)$ by $u_{m\bk}(\mathbf r)$,
in the spirit of the $\bk \cdot {\bf p}$ approximation.

Our choice of calculating the polaron wavefunction within Rayleigh-Schr\"odinger perturbation theory 
is justified since this theory is valid in the range $\alpha \leq 5$ (see Supplementary 
Ref.~\citenum{Mahan}, p.~516), and the largest Fr\"ohlich coupling constant in anatase TiO$_2$ is 
$\alpha_\parallel = 3.0$ (see Supplementary Note~2).

We note that the present many-body approach for calculating the polaron wavefunction differs significantly
from DFT calculations of excess electrons in oxides using large 
supercells\cite{DiValentin2011, Deak2012, Setvin2014, Spreafico2014}.
In fact in the present case we employ a quantum description of both electrons and nuclei, and a 
dynamical (non-adiabatic) description of their interactions. In contrast, supercell DFT calculations
describe the nuclei as classical particles and decouple electronic and nuclear degrees of freedom
using the Born-Oppenheimer adiabatic approximation. These approximations are not justified in the
case of TiO$_2$, and have led to results which are very sensitive to the DFT exchange and correlation
functional and the size of the supercell. Our present perturbative approach does not suffer from these
shortcomings.

\clearpage

\section*{Supplementary References}

%\bibliographystyle{naturemag}
%\bibliography{biblio-tio2}

\begin{thebibliography}{10}
\expandafter\ifx\csname url\endcsname\relax
  \def\url#1{\texttt{#1}}\fi
\expandafter\ifx\csname urlprefix\endcsname\relax\def\urlprefix{URL }\fi
\providecommand{\bibinfo}[2]{#2}
\providecommand{\eprint}[2][]{\url{#2}}

\bibitem{Langreth1970}
\bibinfo{author}{Langreth, D.~C.}
\newblock \bibinfo{title}{Singularities in the {X}-ray spectra of metals}.
\newblock \emph{\bibinfo{journal}{Phys. Rev. B}} \textbf{\bibinfo{volume}{1}},
  \bibinfo{pages}{471--477} (\bibinfo{year}{1970}).

\bibitem{Hedin1980}
\bibinfo{author}{Hedin, L.}
\newblock \bibinfo{title}{Effects of recoil on shake-up spectra in metals}.
\newblock \emph{\bibinfo{journal}{Phys. Scr.}} \textbf{\bibinfo{volume}{21}},
  \bibinfo{pages}{477--480} (\bibinfo{year}{1980}).

\bibitem{Aryasetiawan1996}
\bibinfo{author}{Aryasetiawan, F.}, \bibinfo{author}{Hedin, L.} \&
  \bibinfo{author}{Karlsson, K.}
\newblock \bibinfo{title}{Multiple plasmon satellites in {Na} and {Al} spectral
  functions from \textit{ab initio} cumulant expansion}.
\newblock \emph{\bibinfo{journal}{Phys. Rev. Lett.}}
  \textbf{\bibinfo{volume}{77}}, \bibinfo{pages}{2268--2271}
  (\bibinfo{year}{1996}).

\bibitem{Gumhalter2016}
\bibinfo{author}{Gumhalter, B.}, \bibinfo{author}{Kova\ifmmode~\check{c}\else
  \v{c}\fi{}, V.}, \bibinfo{author}{Caruso, F.}, \bibinfo{author}{Lambert, H.}
  \& \bibinfo{author}{Giustino, F.}
\newblock \bibinfo{title}{On the combined use of {GW} approximation and
  cumulant expansion in the calculations of quasiparticle spectra: The paradigm
  of {Si} valence bands}.
\newblock \emph{\bibinfo{journal}{Phys. Rev. B}} \textbf{\bibinfo{volume}{94}},
  \bibinfo{pages}{035103} (\bibinfo{year}{2016}).

\bibitem{Zhou2015}
\bibinfo{author}{Zhou, J.~S.} \emph{et~al.}
\newblock \bibinfo{title}{Dynamical effects in electron spectroscopy}.
\newblock \emph{\bibinfo{journal}{J. Chem. Phys.}}
  \textbf{\bibinfo{volume}{143}}, \bibinfo{pages}{184109}
  (\bibinfo{year}{2015}).

\bibitem{Berciu2006}
\bibinfo{author}{Berciu, M.}
\newblock \bibinfo{title}{Green's function of a dressed particle}.
\newblock \emph{\bibinfo{journal}{Phys. Rev. Lett.}}
  \textbf{\bibinfo{volume}{97}}, \bibinfo{pages}{036402}
  (\bibinfo{year}{2006}).

\bibitem{Grimvall}
\bibinfo{author}{Grimvall, G.}
\newblock \emph{\bibinfo{title}{The Electron-Phonon Interaction in Metals}}
  (\bibinfo{publisher}{North-Holland}, \bibinfo{address}{New York},
  \bibinfo{year}{1981}).

\bibitem{Moser2013}
\bibinfo{author}{Moser, S.} \emph{et~al.}
\newblock \bibinfo{title}{Tunable polaronic conduction in anatase {TiO$_2$}}.
\newblock \emph{\bibinfo{journal}{Phys. Rev. Lett.}}
  \textbf{\bibinfo{volume}{110}}, \bibinfo{pages}{196403}
  (\bibinfo{year}{2013}).

\bibitem{Hedin1985}
\bibinfo{author}{Bardyszewski, W.} \& \bibinfo{author}{Hedin, L.}
\newblock \bibinfo{title}{A new approach to the theory of photoemission from
  solids}.
\newblock \emph{\bibinfo{journal}{Phys. Scr.}} \textbf{\bibinfo{volume}{32}},
  \bibinfo{pages}{439--450} (\bibinfo{year}{1985}).

\bibitem{Kas2016}
\bibinfo{author}{Kas, J.~J.}, \bibinfo{author}{Rehr, J.~J.} \&
  \bibinfo{author}{Curtis, J.~B.}
\newblock \bibinfo{title}{Particle-hole cumulant approach for inelastic losses
  in x-ray spectra}.
\newblock \emph{\bibinfo{journal}{Phys. Rev. B}} \textbf{\bibinfo{volume}{94}},
  \bibinfo{pages}{035156} (\bibinfo{year}{2016}).

\bibitem{Berger1997}
\bibinfo{author}{Gonzalez, R.~J.}, \bibinfo{author}{Zallen, R.} \&
  \bibinfo{author}{Berger, H.}
\newblock \bibinfo{title}{Infrared reflectivity and lattice fundamentals in
  anatase {TiO$_2$}}.
\newblock \emph{\bibinfo{journal}{Phys. Rev. B}} \textbf{\bibinfo{volume}{55}},
  \bibinfo{pages}{7014--7017} (\bibinfo{year}{1997}).

\bibitem{Verdi2015}
\bibinfo{author}{Verdi, C.} \& \bibinfo{author}{Giustino, F.}
\newblock \bibinfo{title}{Fr\"ohlich electron-phonon vertex from first
  principles}.
\newblock \emph{\bibinfo{journal}{Phys. Rev. Lett.}}
  \textbf{\bibinfo{volume}{115}}, \bibinfo{pages}{176401}
  (\bibinfo{year}{2015}).

\bibitem{Frohlich1954}
\bibinfo{author}{Fr\"ohlich, H.}
\newblock \bibinfo{title}{Electrons in lattice fields}.
\newblock \emph{\bibinfo{journal}{Adv. Phys.}} \textbf{\bibinfo{volume}{3}},
  \bibinfo{pages}{325--361} (\bibinfo{year}{1954}).

\bibitem{Mishchenko2000}
\bibinfo{author}{Mishchenko, A.~S.}, \bibinfo{author}{Prokof'ev, N.~V.},
  \bibinfo{author}{Sakamoto, A.} \& \bibinfo{author}{Svistunov, B.~V.}
\newblock \bibinfo{title}{Diagrammatic quantum {Monte Carlo} study of the
  {Fr\"ohlich} polaron}.
\newblock \emph{\bibinfo{journal}{Phys. Rev. B}} \textbf{\bibinfo{volume}{62}},
  \bibinfo{pages}{6317--6336} (\bibinfo{year}{2000}).

\bibitem{Mahan}
\bibinfo{author}{Mahan, G.~D.}
\newblock \emph{\bibinfo{title}{Many-Particle Physics}}
  (\bibinfo{publisher}{Plenum}, \bibinfo{address}{New York},
  \bibinfo{year}{1981}).

\bibitem{DiValentin2011}
\bibinfo{author}{Di~Valentin, C.} \& \bibinfo{author}{Selloni, A.}
\newblock \bibinfo{title}{Bulk and surface polarons in photoexcited anatase
  {TiO$_2$}}.
\newblock \emph{\bibinfo{journal}{J. Phys. Chem. Lett.}}
  \textbf{\bibinfo{volume}{2}}, \bibinfo{pages}{2223--2228}
  (\bibinfo{year}{2011}).

\bibitem{Deak2012}
\bibinfo{author}{De\'ak, P.}, \bibinfo{author}{Aradi, B.} \&
  \bibinfo{author}{Frauenheim, T.}
\newblock \bibinfo{title}{Quantitative theory of the oxygen vacancy and carrier
  self-trapping in bulk {TiO$_2$}}.
\newblock \emph{\bibinfo{journal}{Phys. Rev. B}} \textbf{\bibinfo{volume}{86}},
  \bibinfo{pages}{195206} (\bibinfo{year}{2012}).

\bibitem{Setvin2014}
\bibinfo{author}{Setvin, M.} \emph{et~al.}
\newblock \bibinfo{title}{Direct view at excess electrons in {TiO$_2$} rutile
  and anatase}.
\newblock \emph{\bibinfo{journal}{Phys. Rev. Lett.}}
  \textbf{\bibinfo{volume}{113}}, \bibinfo{pages}{086402}
  (\bibinfo{year}{2014}).

\bibitem{Spreafico2014}
\bibinfo{author}{Spreafico, C.} \& \bibinfo{author}{VandeVondele, J.}
\newblock \bibinfo{title}{The nature of excess electrons in anatase and rutile
  from hybrid {DFT} and {RPA}}.
\newblock \emph{\bibinfo{journal}{Phys. Chem. Chem. Phys.}}
  \textbf{\bibinfo{volume}{16}}, \bibinfo{pages}{26144--26152}
  (\bibinfo{year}{2014}).

\end{thebibliography}


\end{document}
